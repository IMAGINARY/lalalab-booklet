\epigraph{ \itshape ``Mathematics and music, the most sharply contrasted fields of scientific activity which can be found, and yet related, supporting each other, as if to show forth the secret connection which ties together all the activities of our mind, and which leads us to surmise that the  manifestations of the artist's genius are but the unconscious expressions of a mysteriously acting rationality.''
}
{\vspace{0.5em}Hermann von Helmholtz (1821 - 1894) \\
a \textit{Vorträge und Reden} (1884, 1896), Vol 1, p 122.}

La música i les matemàtiques comparteixen molts atributs com a camps d'estudi. Les dues disciplines estudien objectes abstractes, tenen estructures complexes i les seves pròpies normes per a la seva manipulació, una notació clarament definida, i són absolutament precises als seus resultats. Treballar a les dues requereix pràctica, creativitat i una ment analítica. Vist això, no és cap sorpresa que la música i les matemàtiques estan estretament relacionades.

La seva relació va molt més enllà de les habilitats necessàries per al seu estudi. Les matemàtiques estan profundament lligades a la música, des de les físiques del so fins a la construcció d'instruments, des de patrons rítmics fins a l'harmonia tonal, des de la música clàssica fins a l'electrònica. Les matemàtiques formen la base de la música i el nostre enteniment de l'art de la mateixa manera que sosté la física i el nostre enteniment del món.

Escoltar música amb oïda matemàtica porta a l'amant de la música a una comprensió més profunda, amb apreciació dels detalls, a un major gaudi de l'art; i als professionals l'habilitat de compondre i eines per  expressar la seva creativitat.

\begin{flushright}
IMAGINARY, 2019 %update 2021 ?
\end{flushright}

\vfill
La La Lab - Les Matemàtiques de la Música / Booklet d'exhibició \\
IMAGINARY, \url{www.imaginary.org}, \href{mailto:info@imaginary.org}{info@imaginary.org} \\
Versió 0.5CAT, \today / Llicència: CC BY-SA


\section*{La La Lab - Les Matemàtiques de la Música}

Pensat en col·laboració amb experts en investigació de la música i les matemàtiques, La La Lab combina el format de laboratori amb exposicions interactives. Per aquest motiu podem revelar les impressionants connexions entre les matemàtiques i la música tot impulsant la creativitat musical i el coneixement matemàtic.

La la Lab és una experiència de laboratori modular. Aquest booklet t'ajudarà a donar una ullada més profunda. Hi pots trobar informació complementària sobre cada mòdul, enllaços per descobrir característiques addicionals i lectures recomanades. Si us plau, agafa'n una còpia o descarrega'l del nostre web. l'exposició és open source i està disponible sota llicències gratuïtes. Pots trobar tot el software, imatges, text i dades 3D a: \url{lalalab.imaginary.org}

IMAGINARY és una organització sense ànim de lucre per a la comunicació interactiva, oberta, artística i col·laborativa de les matemàtiques modernes dirigides al públic general. Creat el 2007 al Mathematische Forschungsinstitut Oberwolfach, un institut a Libniz, IMAGINARY ha rebut múltiples guardons per la seva contribució a la divulgació científica. Des del 2008 s'han organitzat més de 340 activitats en més de 60 països i en 30 idiomes, arribant a milions de visitants.


\section*{La La Lab Crèdits}
\footnotesize
Una exposició creada per IMAGINARY.
Presentat per la fundació Heidelberg Laureate Forum Foundation.
El desenvolupament de l'exposició i la producció dels mitjans s'ha fet possible gràcies al suport de la fundació Klaus Tschira Stiftung.

\paragraph{Comitè d'experts i mòduls:}
Moreno Andreatta (CNRS / IRCAM / Sorbonne University and USIAS / IRMA / University of Strasbourg), Manuela Donoso (Brooklyn Research), Thomas Noll (ESMUC Barcelona, TU Berlin), Luisa Pereira (New York University), Jürgen Richter-Gebert (TU München).


\paragraph{Equip IMAGINARY:}
Andreas Matt (Director), Daniel Ramos (Curador, Recerca), Kathrin Unterleitner (Gestió de projectes), Bianca Violet (Contingut, Comunicació), Christian Stussak (Software, Exposicions), Eric Londaits (Software, Exposicions), Sebastián Uribe (Consell tècnic, Exposicions), Malte Westphalen (Escenografia, Disseny), Konrad Renner (Disseny), Tobias Hermann (Hardware), Lukas Reck (Producció, LLum i So), Daniel Weiss (Producció), Antonia Mey (Assistència), Magdalena Hreczynska (Assistència).

\paragraph{Mòduls:}
Corentin Guichaoua (Univ. Strasbourg), Philipp Legner (Mathigon), Patrick Wilson, Aaron Montag and Konrad Heidler (TU
 München), Gerhard Widmer, Stefan Balke, Carlos Eduardo Cancino Chacón and Florian Henkel (JKU Linz), Neil Sloane (OEIS Foundation), Neil Thapen (Academy of Sciences of the Czech Republic), Vitor Guerra Rolla, Pedro Arthur and José Ezequiel Soto Sánchez (Visgraf Lab, Instituto de Matemática Pura e Aplicada, Rio de Janeiro), Ryan Cashman (Nano Animal).

\paragraph{Textos del booklet:}
Daniel Ramos, Sebastián Uribe, Kathrin Unterleitner, Andreas Daniel Matt, Bianca Violet and Eric Londaits (IMAGINARY), Jürgen Richter-Gebert (TU Munich), Pedro Arthur, Vitor Guerra Rolla, and José Ezequiel Soto Sánchez (Visgraf Lab, IMPA), Gerhard Widmer (JKU Linz), Thomas Noll (ESMUC Bar\-ce\-lo\-na, TU Berlin), Ryan Cashman (Nano Animal), Manuela Donoso (Brookly Research), Luisa Pereira (New York University).

\paragraph{Traducció al català:} David Fornell.

\paragraph{Agraïments:}
Heidelberg Laureate Forum Foundation.

\paragraph{Amb el suport de:}
Technische Universität München, Springer Publishing House, The Looking Glass, Taylor \& Francis Publishing House.

\paragraph{Agraïments:}
Alba Málaga Sabogal, Yann Orlarey, Catinca Dumitrascu, Google Magenta Team, Tero Parviainen, Ricardo Dodds, Exploratorium San Francisco, all film authors, European Research Council (ERC), project no.670035.

Els continguts de l'exposició estan disponibles sota llicències open source.
\normalsize
