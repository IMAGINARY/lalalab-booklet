\section{Tonnetz}
Les escales que utilitzem tradicionalment a la música occidental és només una selecció de tons, és a dir, una selecció de freqüències que utilitzem per afinar un instrument per fer música.

Com anomenem les notes? Com les representem gràficament? Quina notació utilitzem per escriure-les en una partitura? La primera resposta, i la més obvia, és que les ordenem d'agudes a greus i les anomenem de forma seqüencial. De tota manera, això no reflecteix les seves relacions o els seus rols al llenguatge musical.

Tocant alhora una primera nota amb una freqüència i una segona amb una freqüència doble (ràtio 1:2), el resultat és un so consonant i tots dos sons es barregen de manera que sembla que siguin la ``mateixa'' nota. A l'afinació moderna (el sistema de temperament igual del piano), una octava es divideix en 12 parts iguals i cadascuna d'aquestes correspon a les 12 notes de l'escala cromàtica.

Però no utilitzem de la mateixa manera aquestes 12 notes, algunes s'utilitzen més sovint, almenys històricament des que la notació es va començar a utilitzar a l'edat mitjana. Per aquest motiu tenim les notes ``principals'' en una escala, que representen les tecles blanques al piano (Do, Re, Mi, Fa, Sol, La, Si, si utilitzem la notació llatina, o C, D, E, F, G, A, B, si utilitzem la notació anglesa) i que anomenem escala diatònica. Per altra banda, trobem unes notes ``secundàries'' que es descriuen com alteracions de les principals amb sostinguts $\sharp$ o bemolls $\flat$  (Com un Do$\sharp$ o un Mi$\flat$ en la notació llatina o com un C$\sharp$ o un E$\flat$ utilitzant la notació anglesa) i que es representen amb les tecles negres.

D'aquesta manera hi ha 7 notes principals en una octava i la vuitena té el mateix nom que la primera (per aquest motiu el terme octava).

Ordenar les notes d'agut a greu, com en un teclat de piano, pot ser raonable, però no reflecteix la relació entre les notes.

A causa de la relació d'octava, té sentit ordenar  les notes en un cercle. Aquest s'anomena el cercle cromàtic. Les notes, utilitzant la notació anglesa, són:

\begin{center}
C, C$\sharp$, D, D$\sharp$, E, F, F$\sharp$, G, G$\sharp$, A, A$\sharp$, B
\end{center}

i després de l'última nota de l'escala (B), continuem amb la primera nota (C). A l'afinació moderna, els intervals són tots iguals (els anomenats semitons) i els sostinguts i bemolls són coincidents, és a dir, C$\sharp$ = D$\flat$. Aquesta relació s'anomena identificació enharmònica.

Una altra relació fonamental és la quinta (o, per ser precisos, la quinta perfecta). Aquest és l'interval entre Do i Sol o entre Fa i Do i que correspon a l'addició de 7 semitons a la nota inicial. S'anomena quinta ja que, a l'escala de 7 notes, el Do és la primera nota i el Sol és la cinquena. Després de l'octava, la quinta és l'interval més consonant, com qualsevol pot sentir tocant les dues notes alhora. Per això, té molt de sentit tenir-les a prop com al Cercle de Quintes, que està fet per la llista de notes en salts de 7 semitons:
\begin{center}
C, G, D, A, E, B, G$\flat$, D$\flat$, A$\flat$, E$\flat$, B$\flat$, F
\end{center}
Com amb l'anterior cercle, F va seguit de C.

Aquesta representació és molt útil pels músics, ja que les notes que encaixen bé hi són representades properes. Això també explica les notes blanques del piano, representades per les notes A, B, C, D, E, F, G, són les   ``principals'' (amb relació a les ``secundàries''): si comences a F, pots repetir la quinta 6 vegades i això et donarà una cadena amb les 7 notes que formen l'escala diatònica. La nota ``C'' és la primera de la notació llatina (Do, Re, Mi, Fa, Sol, La, Si), en el que s'anomena el ``mode major'', però , si comencem l'escala a A, com fa la notació anglesa, obtenim el ``mode menor''.

Dues altres relacions molt consonants són la Tercera Major (afegir 4 semitons) i la Tercera Menor (afegir 3 semitons). Per exemple, entre C i E hi ha una Tercera Major (E és la 3a nota de l'escala). I entre C i E$\flat$, o entre D i F, hi ha una Tercera Menor (només hi ha una distància de 3 semitons).

Quan toquem diverses notes alhora obtenim el que s'anomena un acord. Els acords més comuns són les tríades (en què es toquen 3 notes alhora) i resulta que la recepta bàsica per fer un acord major és triar una nota x (la tònica) i afefir-ne la tercera major i la quinta. Això ens dóna 3 notes (vist en semitons: x, x+4, x+7) que sonen molt brillants i agradables a l'oïda. Per exemple, la tríada C - E - G és l'acord de C major. Anàlogament, els acords menors es creen amb la tònica x, la seva quinta i la seva tercera menor (en semitons, x, x+3, x+7). Aquests acords sonen més tristos i obscurs. Per exemple, C - E$\flat$ - G és l'acord de C menor.

Hi ha una representació gràfica de les notes que reflecteix algunes d'aquestes relacions. Es remunta al matemàtic Leonhard Euler i és coneix avui dia com a Tonnetz (El mot alemany per ``xarxa de tons''). El Tonnetz clàssic (que anomenem 3,4,5) consisteix d'una graella triangular on cada vèrtex s'associa a una nota (fins a una octava). Hi ha 3 línies o direccions en aquesta graella. En una direcció les notes escalen en quintes, en una altra en terceres majors i a l'última en terceres menors. Això és possible gràcies a que pujar 7 semitons en una direcció es correspon a pujar-ne 4 semitons i 3 semitons als altres dos costats de cada triangle. Les tres notes formen un triangle al cercle cromàtic amb costats 3, 4 i 5, que és l'etiqueta que li donem a aquest Tonnetz.

Amb aquest diagrama, totes les notes separades per una quinta, una tercera major o una tercera menor són adjacents. I, com si fos poc, tots els acords majors i menors es poden representar amb les cares triangulars d'aquest diagrama.

També hi ha altres tipus de triades, depenent dels intervals entre les seves notes. Per exemple, els acords augmentats són (x, x+4, x+8) formarien un triangle de costats 4,4,4 al cercle cromàtic (o al de quintes). Podem construir un Tonnetz que representi aquests acords utilitzant aquests intervals com a passos incrementals a cada direcció. Qualsevol grup de tres nombres a, b, c, que sumin 12 representa un triangle al cercle cromàtic. Fent servir combinatòria, hi ha 12 possibles maneres de triar 3 nombres a, b, i c, entre 1 i 12, de manera que a+b+c=12. Aquests són els 12 tonnetz possibles.

Al diagrama també podem veure els dobles grafs del tonnetz. Un graf dual es construeix substituint cada cara per un vèrtex, cada vèrtex per una cara, i vèrtexs connectats amb cares adjacents. Aquest diagrama té la mateixa informació que l'original, però ara cada tríada és un vèrtex d'una figura hexagonal i cada nota n'és una cara.

La representació amb Tonnetz conté un munt d'informació musical. Per exemple, els triangles adjacents representen triades amb dos tons en comú, donant lloc a  progressions d'acords naturals si es viatja d'un a l'altre.

Per acabar, aquestes representacions gràfiques es poden utilitzar per transformar una peça. La transformació més bàsica és el canvi de to, afegint una quantitat donada de semitons a totes les notes d'una peça musical. Això és una pràctica usual per adaptar les peces al rang vocàlic dels cantants. Geomètricament correspon a la translació al piano, però les tecles blanques i negres ja no coincideixen. És millor considerar una rotació al cercle cromàtic, de manera que les notes només queden rotades, o la translació al Tonnetz.

Un exemple d'aquest tipus de transformacions seria una rotació al cercle de quintes. Això només són salts de 7 semitons (pel fet que una quinta eren 7 semitons). Altra vegada, si tenim la peça representada al Tonnetz, això es correspon a una translació de la xarxa en qualsevol direcció.

Una transformació més interessant és la rotació del Tonnetz. Pensa en una peça representada al Tonnetz. Fixa-hi una nota (per exemple la primera de la peça), i després canvia la resta de notes per la seva contrapart després de girar el Tonnetz 180 graus. Aquesta transformació s'anomena ``harmonia negativa'' i transforma tots els acords majors a menors i viceversa. Pots escoltar-ne un exemple a l'exposició, seleccionant el tonnetz 3,4,5 clàssic.

\begin{sectcredits}
\item[Autors del mòdul:] Moreno Andreatta i Corentin Guichaoua (SMIR Project). Adaptat per Philipp Legner.
\item[Agraïments:] Amb el suport de CNRS/IRCAM/Sorbonne University, USIAS (University of Strasbourg Institute for Advanced Study), IRMA/University of Strasbourg.
\item[Text:] Daniel Ramos (IMAGINARY).
\end{sectcredits}
