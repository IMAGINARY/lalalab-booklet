\section{La La Cinema}
El La La Cinema és una col·lecció de videos curts centrats en la relació entre la música i les matemàtiques. La cartellera inclou pelicules, videos i animacions que s'han creat amb finalitats educacionals, corporatives i artístiques, o per diversió. La durada total de la filmoteca és d'al voltant d'una hora (1:05h). Agraïm a tots els autors per donar-nos el permis per projectar les seves pel·lícules a la nostra exposició.

La llista completa és la següent:

\subsubsection*{Peace for Triple Piano (4:15)}
De Vi Hart en col·laboració amb Henry Segerman i amb l'ajuda addicional de Sabetta Matsumoto. \\
Es tracta d'un vídeo esfèric en un espai matemàticament triplificat amb simetria espai-temps.
This is a spherical video in a mathematically triplified space with symmetry in space-time.

\subsubsection*{Making of Peace for Triple Piano (9:42)}
De Vi Hart en col·laboració amb Henry Segerman i amb l'ajuda addicional de Sabetta Matsumoto\\
Aquest vídeo explica tan els conceptes com les mates utilitzats per crear la pel·lícula anterior (Peace for Triple Piano).

\subsubsection*{J.S. Bach - Crab Canon on a Möbius Strip (3:07)}
De Jos Leys - www.josleys.com \\
El manuscrit retrata una seqüència musical pensada per tocar-se en els dos sentits.

La La Relation: Compara aquest vídeo amb el Canon de Bach del mòdul ``Show me the music''.

\subsubsection*{Improvising a canon at the fifth above (4:22)}
Cantants: Peter Schubert, Schulich School of Music, McGill University, Montreal, i Dawn Bailey\\
Producció: Tuscan Bean Soup, Montreal\\
Productor: George Massenburg\\
Editor: Michelle Hugill\\
Concepte i estratègia: Shelley Stein-Sacks\\
Post-producció i edició de so: David Rafferty\\
Peter Schubert i Dawn Bailey mostren com improvisar un canon amb estil renaixentista.

La La Relation: Compara aquest vídeo amb el Canon de Bach del mòdul ``Show me the music''.

\subsubsection*{Algebraic Vibrations (2:48)}
Bianca Violet i Stephan Klaus\\

En aquest film s'aproximen diferents patrons de vibració característics de cambors aconseguint simular-los amb superfícies algebraiques.

La La Relation: El vídeo té relació amb el mòdul ``Fingerprint of Sound''.

\subsubsection*{Why It's Impossible to Tune a Piano (4:19)}
Henry Reich - minutephysics\\

Aquest vide explica el motiu pel qual és matemàticament impossible afinar totes les tecles d'un piano de forma consistent utilitzant harmònics.

La La Relation: Aquest vídeo té relació amb el mòdul ``Scale Lab''.

\subsubsection*{Dance of the Line Riders (2:13)}
Animació: Mark Robbins - DoodleChaos \\
Música: Dance of the Sugar Plum Fairy Kevin MacLeod (incompetech.com) CC BY 3.0

L'autor va sincronitzar la cançó ``Dance of the Sugar Plum Fairy'' de Tchaikovsky amb una cançó de line rider (\url{www.linerider.com}). Ell mateix va dibuixar a mà tot el que aparèix al vídeo.

\subsubsection*{CYMATICS: Science Vs. Music (5:52)}
Nigel John Stanford \\
Música: de l'àlbum Solar Echoes\\

Aquest vídeo conté àudio visualitzat amb experiments científics - inclòs el Plat e Chladni, el Tub de Ruben, la bobina de Tesla i fluid ferromagnètic. Tots els experiments són reals.

La La Relation: Aquest vídeo té relació amb el mòdul ``Fingerprint of Sound''.

\subsubsection*{Four Clarinets (3:49)}
Animació: Jeffrey Ventrella \\
Música: Four Clarinets de Robby Elfman, tocat per músics del USC Thornton School of Music. \\

Aquest vídeo mostra una animació artística de la partitura creada per un algorisme juntament amb alguns ajusts que l'autor hi feia a temps real mentre n'escoltava la música.

\subsubsection*{Gerhard Widmer on Expressive Music Performance (5:04)}
Director: Ethan Vincent \\
Producció: Austrian Science Fund (FWF) \url{www.fwf.ac.at/en/} \\
Gravat al Bösendorfer Piano Factory, Wiener Neustadt, Austria. \\

Gerhard Widmer, un investigador del camp de l'intel·ligència artificial i la música, parla sobre la seva recerca sobre l'actuació expressiva de piano, el rol del concert monitoritzat per ordinador del sistema CEUS de Boesendorfer en aquest procés, i després controla el tempo i dinàmiques d'una actuació de Chopin feta pel mateix sistema CEUS de Boesendorfer fent servir una mà i un Theremin amb connexió MIDI.

La La Relation: Aquest vídeo té relació amb el mòdul ``Con Espressione!''.

\subsubsection*{Dynamic real-time MRI Bruder Jakob/Frère Jacques (0:39)}
Del DVD-ROM ``The Voice'', © Helbling Verlag GmbH, Esslingen / Freiburger Institut fur Musikermedizin (FIM)\\

En aquest vídeo s'utilitza una ressonància magnètica per mostrar amb detall l'interior del cos mentre es canta.

La La Relation: Aquest vídeo té relació amb el mòdul ``Pink Trombone''.

\subsubsection*{Muse - Take A Bow (4:35)}
Animació: Louis Bigo \\
Música: ``Take A Bow'' de l'àlbum ``Black Holes and Revelations'' composat per Matthew Bellamy i tocat per Muse. \\

Aquest vídeo presenta l'anàlisi harmònic de la cançó Take A Bow. Els acords es representen a l'espai tonal anomenat Tonnetz, que mostra els tons musicals en eixos associats als intervals de quinta (horitzontal), i les terceres menors i majors (diagonals).

La La Relation: Aquest vídeo té relació amb el mòdul ``Tonnetz''.

\subsubsection*{La Sera (ZhiZhu) (4:11)}
Animació: Gilles Baroin\\
Música i veus: Moreno Andreatta\\
Lletra: Mario Luzi\\

La teranyina simbolitza el centr tonal i s'il·lustren els moviments harmònics bàsics: quintes descendents i la seva relació relativa. A la composició per La Sera, Moreno Andreatta utilitza un camí Hamiltonià i el principi d'atracció tonal.

La La Relation: Aquest vídeo té relació amb el mòdul ``Tonnetz''.

\subsubsection*{The Science Behind the Arts: The Maths Behind Music (3:51)}
University of Surrey\\

Què determina la freqüència d'una corda que vibra? Que és un interval musical?

La La Relation: Aquest vídeo té relació amb el mòdul ``Scale Lab''.

\subsubsection*{JSBach333 canone permutativo al triangolo (5:46)}
Producció i animació: Ulrich Seidel - seidel.graphics\\
Compositor i director: Thomas M. J. Schäfer\\
Músics del Fellbacher Kammerorchester: Regine Rosin (Violí), Daniel Egger (Viola), Cora Wacker (Violoncel) \\

Es tracta d'una animació d'un dels canons presentats a la competició Bach333 del 2018:\url{https://seidel.graphics/bach333en/}. El canon tonal i modern es refereix al nom, l'obra i el contrapunt de l'execució de J. S. Bach.

La La Relation: Compara aquest vídeo amb el canon de Bach del mòdul ``Show me Music''.

\begin{center} \large * * * \end{center}
\vspace{1em}

Fora del programa oficial del La La Cinema, hi ha alguns films que recomanem que vegis. Aquests no es mostren a l'exposició, ja que són massa llargs o no tenim els drets per projectar-los públicament.

\subsubsection*{Music And Measure Theory (13:12)}
Grant Sanderson - 3blue1brown

\subsubsection*{Visual Fourier Transform (20:56)}
Grant Sanderson - 3blue1brown

\subsubsection*{Musician Explains One Concept in 5 Levels of Difficulty (15:41)}
Jacob Collier \& Herbie Hancock | WIRED

\subsubsection*{Visualizing the Notes as Ratios (11:11)}
Why These Notes - Adventures in Music Theory

\subsubsection*{Poetry, Daisies and Cobras: Math class with Manjul Bhargava (11:42)}
NDTV

\subsubsection*{A different way to visualize rhythm (5:22)}
John Varney, TEDed

\subsubsection*{Quadruple Major/Minor Canon (2:30)}
PlayTheMind

\subsubsection*{Fractal Fugues (self-similar counterpoint) (3:01)}
Jeffrey Ventrella

\subsubsection*{Illustrated Music (youtube channel)}
Tom Johnson

\subsubsection*{Debussy, Arabesque \#1, Piano Solo (5:04)}
Stephen Malinowski

\subsubsection*{Singing in the MRI with Tyley Ross - Making the Voice Visible (4:14)}
Tyley Ross

\subsubsection*{MatheMusic4D (youtube channel)}
Gilles Baroin

\subsubsection*{1ucasvb (youtube channel)}
Lucas Vieira

\begin{sectcredits}
\item[Sel·lecció de vídeos:] Bianca Violet (IMAGINARY).
\item[Agraïments:] Gràcies a tots els autors dels vídeos per la seva contribució.
\end{sectcredits}
