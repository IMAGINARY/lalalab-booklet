\epigraph{ \itshape ``Mathematics and music, the most sharply contrasted fields of scientific activity which can be found, and yet related, supporting each other, as if to show forth the secret connection which ties together all the activities of our mind, and which leads us to surmise that the  manifestations of the artist's genius are but the unconscious expressions of a mysteriously acting rationality.''
}
{\vspace{0.5em}Hermann von Helmholtz (1821 - 1894) \\
in \textit{Vorträge und Reden} (1884, 1896), Vol 1, p 122.}


Music and Mathematics share many similarities as fields of study. Both disciplines study abstract objects, have complex structures and manipulation rules, a well defined notation, and are absolutely precise in their results. Working on them requires practice, creativity and an analytical mind. It is no surprise that Mathematics and Music are closely related.

But their relationship goes far beyond the skills needed for its study. Mathematics is deeply infused in all aspects of Music, from the physics of sound to the crafting of instruments, from rhythmic patterns to tonal harmony, from classical to electronic music. Mathematics supports music and our understanding of art the same way as it supports physics and our understanding of the world.

Hearing music with mathematical ears brings to the music lover deeper comprehension, appreciation for details, and a greater enjoyment of the art; and to professionals the ability to compose and tools to express their creativity.

\begin{flushright}
IMAGINARY, 2019
\end{flushright}

\vfill
La La Lab - The Mathematics of Music / Exhibition Booklet \\
IMAGINARY, \url{www.imaginary.org}, \href{mailto:info@imaginary.org}{info@imaginary.org} \\
Version 0.5, \today / License: CC BY-SA


\section*{La La Lab - The Mathematics of Music}

Conceived in collaboration with experts in current research of music and mathematics, La La Lab combines a laboratory format with interactive exhibits. Therefore we can reveal the stunning connections between mathematics and music, pushing the boundaries of musical creativity and mathematical knowledge.

La La Lab is a modular lab exhibition experience. This booklet invites you to dive deeper. You will find exhibits and accompanying information, links to discover additional features and recommended readings. Please take a copy with you or download it on our web page. This exhibition is open source and available under free licenses. You can find all software, images, texts, 3d data at:  \url{lalalab.imaginary.org}


IMAGINARY is a non-profit company for interactive and open, artistic and collaborative communication of modern mathematics to the general public. Created in 2007 at the Mathematische Forschungsinstitut Oberwolfach, a Leibniz institute, IMAGINARY has received several awards for its contribution to science communication. Since 2008, more than 340 exhibition activities have been organized in more than 60 countries and in 30 languages, attracting millions of visitors.


\section*{La La Lab Credits}
\footnotesize
An exhibition by IMAGINARY.
Presented by the Heidelberg Laureate Forum Foundation.
Development of exhibits and production of exhibition has been made possible by support of the Klaus Tschira Stiftung.

\paragraph{Experts panel and exhibits:}
Moreno Andreatta (CNRS / IRCAM / Sorbonne University and USIAS / IRMA / University of Strasbourg), Manuela Donoso (Brooklyn Research), Thomas Noll (ESMUC Barcelona, TU Berlin), Luisa Pereira (New York University), Jürgen Richter-Gebert (TU München).

\paragraph{IMAGINARY team:}
Andreas Matt (Director), Daniel Ramos (Curation, Research), Kathrin Unterleitner (Project Management), Bianca Violet (Content, Communication), Christian Stussak (Software, Exhibits), Eric Londaits (Software, Exhibits), Sebastián Uribe (Technical Advice, Exhibits), Malte Westphalen (Scenography, Design), Konrad Renner (Design), Tobias Hermann (Hardware), Lukas Reck (Production, Light \& Sound), Daniel Weiss (Production), Antonia Mey (Assistance), Magdalena Hreczynska (Assistance).

\paragraph{Exhibits:}
Corentin Guichaoua (Univ. Strasbourg), Philipp Legner (Mathigon), Patrick Wilson, Aaron Montag and Konrad Heidler (TU
 München), Gerhard Widmer, Stefan Balke, Carlos Eduardo Cancino Chacón and Florian Henkel (JKU Linz), Neil Sloane (OEIS Foundation), Neil Thapen (Academy of Sciences of the Czech Republic), Vitor Guerra Rolla, Pedro Arthur and José Ezequiel Soto Sánchez (Visgraf Lab, Instituto de Matemática Pura e Aplicada, Rio de Janeiro), Ryan Cashman (Nano Animal).

\paragraph{Booklet Texts:}
Daniel Ramos, Sebastián Uribe, Kathrin Unterleitner, Andreas Daniel Matt, Bianca Violet and Eric Londaits (IMAGINARY), Jürgen Richter-Gebert (TU Munich), Pedro Arthur, Vitor Guerra Rolla, and José Ezequiel Soto Sánchez (Visgraf Lab, IMPA), Gerhard Widmer (JKU Linz), Thomas Noll (ESMUC Bar\-ce\-lo\-na, TU Berlin), Ryan Cashman (Nano Animal), Manuela Donoso (Brookly Research), Luisa Pereira (New York University).

\paragraph{With special thanks to:}
Heidelberg Laureate Forum Foundation.

\paragraph{Supported by:}
Technische Universität München, Springer Publishing House, The Looking Glass, Taylor \& Francis Publishing House.

\paragraph{Acknowledgements:}
Alba Málaga Sabogal, Yann Orlarey, Catinca Dumitrascu, Google Magenta Team, Tero Parviainen, Ricardo Dodds, Exploratorium San Francisco, all film authors, European Research Council (ERC), project no.670035.


The exhibition content is available under open source licenses.
\normalsize
