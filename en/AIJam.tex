\section{AI Jam}
Machines can generate and play music, but how good are they at making music together with humans? Could they be part of a Jam Session? Could an Artificial Intelligence eventually replace a missing musician during a band rehearsal?

Even leaving aside questions of musical quality, there are many issues that arise when attempting to compose music using machine learning. One of them, is the generation of a melody. Humans are usually good at producing long-term structures, but machine learning algorithms are not, so the music they produce might sound good on a note-to-note basis, but its structure will seem random and wandering after some bars. AI Jam tackles this problem by explicitly training its Recurrent Neural Network \footnote{\url{www.en.wikipedia.org/wiki/Recurrent_neural_network}} (RNN) with musical structure in mind. When feeding it with thousands of musical pieces, the input was specially labeled whenever a bar was repeating the one immediately preceding it, or the one before that. This is called a Lookback \footnote{\url{www.magenta.tensorflow.org/2016/07/15/lookback-rnn-attention-rnn}} RNN by its developers, the Google Magenta team.

A second technique used is that of Attention. In this case, whenever the RNN produces some output, it looks back at the previous n outputs (with n being configurable), weights them using a calculated attention mask and adds the result. The result is basically the previous n outputs combined, but each given a different amount of emphasis. It is then combined with the output of the current step. In this way, each step produces an output that is related to the previous ones.

By using both mechanisms together, the algorithm can not only produce short fragments of notes that sound good by themselves, but that also form a nice sounding melody when played after a fragment played by a human.

\begin{sectcredits}


\item[Author of the exhibit:] Sebastian Uribe (Exhibit conception and project management), Eric Londaits (Software development), Christian Stussak (Additional software development and OS configuration), and Daniel Weiss (Case design and construction) for IMAGINARY.
\item[Original software (based on):] Yotam Mann, the Magenta and Creative Lab teams at Google.
\end{sectcredits}
