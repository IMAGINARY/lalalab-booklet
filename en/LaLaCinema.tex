\section{La La Cinema}

The film station is a collection of short videos focusing on the relation between music and mathematics. The program includes films, videos and animations that have been created for educational, corporate and artistic purposes, or just for fun. The total length of all films is about one hour (1:05 h). We thank all the authors for giving us permission to screen their films within the La La Lab exhibition.

The following films are shown:

\subsubsection*{Peace for Triple Piano (4:15)}
Vi Hart in collaboration with Henry Segerman, with additional help from Sabetta Matsumoto. \\
This is a spherical video in a mathematically triplified space with symmetry in space-time.

\subsubsection*{Making of Peace for Triple Piano (9:42)}
Vi Hart in collaboration with Henry Segerman, with additional help from Sabetta Matsumoto.\\
This video explains the concepts, as well as the math, movie, and piano magic used to create the previous film (Peace for Triple Piano).

\subsubsection*{J.S. Bach - Crab Canon on a Möbius Strip (3:07)}
Jos Leys - www.josleys.com \\
The manuscript depicts a single musical sequence that is to be played front to back and back to front.

La La Relation: Compare this video with the Bach canon in the exhibit `Show me Music'.

\subsubsection*{Improvising a canon at the fifth above (4:22)}
Singers: Peter Schubert, Schulich School of Music, McGill University, Montreal, and Dawn Bailey\\
Production: Tuscan Bean Soup, Montreal\\
Producer: George Massenburg\\
Editor: Michelle Hugill\\
Concept and strategy: Shelley Stein-Sacks\\
Post-production sound editing: David Rafferty\\
Peter Schubert and Dawn Bailey show how to improvise a canon, Renaissance style. 

La La Relation: Compare this video with the Bach canon in the exhibit `Show me Music'.

\subsubsection*{Algebraic Vibrations (2:48)}
Bianca Violet, Stephan Klaus\\

In this film, different characteristic patterns of a drum vibration are approximated and effectively simulated using algebraic surfaces.
	
La La Relation: This video is related to the exhibit `Fingerprint of Sound'.

\subsubsection*{Why It's Impossible to Tune a Piano (4:19)}
Henry Reich - minutephysics\\

This video explains, why it is mathematically impossible to tune a piano consistently across all keys using harmonics.

La La Relation: This video is related to the exhibit `Scale Lab'.

\subsubsection*{Dance of the Line Riders (2:13)}
Animation: Mark Robbins - DoodleChaos \\
Music: Dance of the Sugar Plum Fairy Kevin MacLeod (incompetech.com) CC BY 3.0

The author synchronized the song ``Dance of the Sugar Plum Fairy'' by Tchaikovsky to a line rider track (\url{www.linerider.com}). He drew everything by hand.

\subsubsection*{CYMATICS: Science Vs. Music (5:52)}
Nigel John Stanford \\
Music: From the album Solar Echoes\\

This video features audio visualized by science experiments - including the Chladni Plate, Ruben's Tube, Tesla Coil and Ferro Fluid. All of the experiments are real.

La La Relation: This video is related to the exhibit `Fingerprint of Sound'.

\subsubsection*{Four Clarinets (3:49)}
Animation: Jeffrey Ventrella \\
Music: Four Clarinets by Robby Elfman, performed by musicians of the USC Thornton School of Music. \\

The video shows an artistic animation of the musical score created by a custom made algorithm, together with some real time adjustments that the author performed while listening to the music. 

\subsubsection*{Gerhard Widmer on Expressive Music Performance (5:04)}
Video director: Ethan Vincent \\
Production: Austrian Science Fund (FWF) \url{www.fwf.ac.at/en/} \\
Filmed at the Bösendorfer Piano Factory, Wiener Neustadt, Austria. \\

Artificial Intelligence \& Music researcher Gerhard Widmer talks about his computer-based research on expressive piano performance, the role of the Boesendorfer computer-monitored concert grand piano CEUS in this process, and then controls (tempo and dynamics) of a Chopin performance on the Boesendorfer CEUS with his hand, using a MIDI Theremin as a control device. 

La La Relation: This video is related to the exhibit `Con Espressione!'.

\subsubsection*{Dynamic real-time MRI Bruder Jakob/Frère Jacques (0:39)}
From the DVD-ROM ``The Voice'', © Helbling Verlag GmbH, Esslingen / Freiburger Institut f¸r Musikermedizin (FIM)\\

In this video, Magnetic resonance imaging (MRI) is used to show detailed images of the inside of the body while singing.

La La Relation: This video is related to the exhibit `Pink Trombone'.

\subsubsection*{Muse - Take A Bow (4:35)}
Animation: Louis Bigo \\
Music: ``Take A Bow'' from the album ``Black Holes and Revelations'' composed by Matthew Bellamy performed by Muse. \\

This video presents the harmonic analysis of the song Take A Bow. Chords are represented within the pitch space named as Tonnetz, which displays musical pitches along axis associated with the intervals of fifth (horizontal), minor and major thirds (diagonals).

La La Relation: This video is related to the exhibit `Tonnetz'.

\subsubsection*{La Sera (ZhiZhu) (4:11)}
Animation:Gilles Baroin\\
Music and vocals: Moreno Andreatta\\
Lyrics: Mario Luzi\\

The spider web symbolizes the tonal center, basic harmonic movements are illustrated: falling fifths and relative relation. In the composition for La Sera, Moreno Andreatta uses both Tonal attraction principia and an Hamiltonian path. 

La La Relation: This video is related to the exhibit `Tonnetz'.

\subsubsection*{The Science Behind the Arts: The Maths Behind Music (3:51)}
University of Surrey\\

What determines the frequency of a vibrating string? What is a musical interval?

La La Relation: This video is related to the exhibit `Scale Lab'.

\subsubsection*{JSBach333 canone permutativo al triangolo (5:46)}
Production, animation: Ulrich Seidel - seidel.graphics\\
Composer and musical director: Thomas M. J. Schäfer\\ 
Musicians of Fellbacher Kammerorchester: Regine Rosin (Violine), Daniel Egger (Viola), Cora Wacker (Violoncello) \\

This is an animation of one of the submitted canons to the Bach333 canon competition contest 2018: \url{https://seidel.graphics/bach333en/}. The tonal and modern canon refers to the name, the oeuvre and the contrapuntal execution of J. S. Bach.

La La Relation: Compare this video with the Bach canon in the exhibit `Show me Music'.


Outside the official La La Cinema program, there are a few films we recommend to watch. They are not screened within the exhibition either because they are two long or we do not have the rights to screen them publicly.

\subsubsection*{Music And Measure Theory (13:12)}
Grant Sanderson - 3blue1brown

\subsubsection*{Visual Fourier Transform (20:56)}
Grant Sanderson - 3blue1brown

\subsubsection*{Musician Explains One Concept in 5 Levels of Difficulty (15:41)}
Jacob Collier \& Herbie Hancock | WIRED

\subsubsection*{Visualizing the Notes as Ratios (11:11)}
Why These Notes - Adventures in Music Theory

\subsubsection*{Poetry, Daisies and Cobras: Math class with Manjul Bhargava (11:42)}
NDTV

\subsubsection*{A different way to visualize rhythm (5:22)}
John Varney, TEDed

\subsubsection*{Quadruple Major/Minor Canon (2:30)}
PlayTheMind

\subsubsection*{Fractal Fugues (self-similar counterpoint) (3:01)}
Jeffrey Ventrella

\subsubsection*{Illustrated Music (youtube channel)}
Tom Johnson

\subsubsection*{Debussy, Arabesque \#1, Piano Solo (5:04)}
Stephen Malinowski

\subsubsection*{Singing in the MRI with Tyley Ross - Making the Voice Visible (4:14)}
Tyley Ross

\subsubsection*{MatheMusic4D (youtube channel)}
Gilles Baroin

\subsubsection*{1ucasvb (youtube channel)}
Lucas Vieira

\vfill

Film selection and curation: Bianca Violet (IMAGINARY).\\
THANKS TO ALL FILM AUTHORS FOR THEIR CONTRIBUTION.


