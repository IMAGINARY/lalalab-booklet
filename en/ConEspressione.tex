\section{Con Espressione!}
A piece of music can be described with mathematical accuracy on a score, with symbols perfectly defined in a well-founded theory. However, music is above all an art, and as such it conveys emotions, feelings, and human sensations. How is it possible to express these feelings, given such a strict notation? How can a performer make a piece come alive? What lies beneath the score sheet and the execution of a piece?

A human performer does not play all the notes of a chord at the same time, nor do they keep a strict tempo; some notes start a few milliseconds earlier, or are released a few milliseconds later, some are played louder than others, some appear quicker, etc. All these nuances allow a performing musician to express themselves and imprint some feelings and emotions onto the performance. A musician is not a machine that perfectly reproduces the score, and these ``imperfections'' or deviations are what make the music alive, something that humans do naturally but a machine could never do... or could it?

This exhibit allows you to explore the difference between a mechanical reproduction of a piece and a more ``human'' interpretation produced by an Artificial Intelligence that was trained to behave as a musician. The visitor takes the role of a music conductor, controlling overall tempo and loudness of a piano performance. Via a camera sensor, the hand of the visitor is tracked in space. The up-down position determines the loudness (volume) of the music, and the left-right position determines the tempo (the speed) of the music. Initially, this is achieved by directly adapting the loudness and tempo of the overall piece according to the position of the hand, but even if the machine obeys you to set these values, the music feels automatic and soul-less. This is because with your hand movements, you can only control overall tempo and loudness, but not the fine details of a performance (such as how to play the individual notes, or how to stress the melody line).

Then, a slider allows you to activate the Artificial Intelligence. The higher the value, the more freedom has the machine to choose small deviations from the prescribed parameters. The machine adjusts the tempo and loudness to be slightly different from what you conduct, to make the music more lively and less ``mechanical''. It also introduces changes in the dynamic spread, micro-timing, and articulation.

\begin{itemize}
\item The loudness is the volume, i.e. the amount of amplification of the sound of each note. Raising the loudness is the most obvious way to stress a note.

\item Dynamic spread relates to loudness differences between notes simultaneously played (e.g., in a chord). This is important to make the melody line come out clearly and to change the overall ``sound'' of a chord.

\item Musical tempo is defined as the rate at which musical events or beats are played. Music performers continually change the tempo, speeding up or slowing down, to express the ``ebb and flow'' of the music. This is what makes music sound natural to us (and we may not even be conscious of all these tempo fluctuations).

\item Microtiming refers to the moment that a note plays with respect to its supposed onset. For example, if a chord consists of several notes that are supposed to be played together, one can advance one note over another by a few milliseconds, so that not all of them are perfectly synchronized. This is inevitable in real-life performance, and it makes the piece more warm, human and expressive.

\item Articulation here refers to the duration of a note with respect to its supposed duration according to the score. Notes can be played a bit longer or shorter than the composer described in the score, tying them together or separating them, which helps to stress or diffuse some notes amongst the others. In musical language, this is described with terms as legato and staccato.
\end{itemize}

Each performer has their own experience, understanding of a piece, and expressive intentions, and communicating these in a performance requires control over musical parameters at many levels -- from precise note-level details like articulation or micro-timing to high-level, long-term tempo and the shaping of its dynamics. The computer program behind this exhibit was trained with hundreds of real-life performances of music pieces to analyze and learn how these parameters are used and controlled in real-life interpretations by human pianists. Experimental results show that computers are already very good at learning the low-level, detailed decisions, but still have problems understanding the larger-scale form and dramatic structure of music, and the high-level shaping this requires. Thus, the exhibit explores and demonstrates a compromise: you control overall loudness and tempo with your hand, at a high level, based on your understanding of the music, and the computer adds its own local details and deviations. In this way, the resulting performance is the product of a true cooperation between a human (you) and a computer (AI).

\begin{sectcredits}

\item[Authors of the exhibit:] Gerhard Widmer, Florian Henkel, Carlos Eduardo Cancino Chacón, Stefan Balke (Institute of Computational Perception, Johannes Kepler University Linz, Austria, Austrian Research Institute for Artificial Intelligence (OFAI), Vienna, Austria), and Christian Stussak, Eric LONDAITS (IMAGINARY).

\item[Text:] Daniel Ramos (IMAGINARY).

\item[Acknowledgments:] This project has received funding from the European Research Council (ERC) under the European Union's Horizon 2020 research and innovation programme (grant agreement number 670035).

\item[References:] \strut
\noindent \begin{itemize}[leftmargin=*]
\item Gerhard Widmer (2017). \emph{Getting Closer to the Essence of Music: The Con Espressione Manifesto}. ACM Transactions on Intelligent Systems and Technology (TIST) 8 (2), 19. \\
\url{www.arxiv.org/pdf/1611.09733.pdf}

\item Carlos Eduardo Cancino-Chacón (2018). \emph{Computational Modeling of Expressive Music Performance with Linear and Non-linear Basis Function Models}. PhD Thesis, Johannes Kepler University Linz (JKU).\\
\url{www.cp.jku.at/research/papers/Cancino_Dissertation_2018.pdf}

\item Carlos E. Cancino-Chacón, Maarten Grachten, Werner Goebl and Gerhard Widmer (2018). \emph{Computational Models of Expressive Music Performance: A Comprehensive and Critical Review}. Frontiers in Digital Humanities 5, Oct. 2018. \\
\url{www.frontiersin.org/articles/10.3389/fdigh.2018.00025/full}
\end{itemize}
\end{sectcredits}
